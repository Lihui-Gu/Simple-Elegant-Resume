\cvsection{研究经历}

\begin{cventries}
% \vspace{-1.0mm}

\cvexperience
{\entrylocationstyle{博士研究生},鲍玉昆教授,管理学院,华中科技大学}
{03/2018 - PRESENT}
{
    \begin{cvitems}
    % \item {提出一种具备收敛性的随机权重卷积神经网络,高效建模进行精准时间序列预测(Zhang et al. Neurocomputing2021)}
    \item {提出一种基于误差反馈的随机权重卷积神经网络预测模型构造方法(Zhang et al. Neurocomputing 2021)}
    \item {提出一种普适逼近的随机权重卷积循环神经网络预测模型构造方法(Working on)}
    \item {提出一种LSTM电力负荷预测模型的特征结构优化策略(Yang et al. CIEEC 2021)}
    \item {提出一种深度神经网络预测模型输入结构的二阶段优化策略(Working on)}
    \item {不同解码结构对深度神经网络预测模型预测精度的影响研究(Working on)}
    \end{cvitems}
}




\cvexperience
{\entrylocationstyle{硕士研究生},蔡淑琴教授,管理学院,华中科技大学}
{09/2015 - 03/2018}
{
    \begin{cvitems}
    \item {提出一种基于机器翻译的会计分录机器编制方法(硕士学位论文)}
    \item {提出一种基于SVM的在线负面口碑识别方法(张心泽等,统计与决策 2017)}
    \end{cvitems}
}

\cvexperience
{\entrylocationstyle{研究助理},何琨教授,计算机科学与技术学院,华中科技大学}
{03/2018 - PRESENT}
{
    \begin{cvitems}
    \item {提出机器翻译对抗攻击的公理定义,成功进行主流机器翻译模型的黑盒攻击(Zhang et al. ACL 2021)}
    \item {提出机器翻译对抗攻击的防御方法,增强主流机器翻译模型的鲁棒性与准确性(Working on)}
    \end{cvitems}
}


\end{cventries}